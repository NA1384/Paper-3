\documentclass[11pt,addpoints,answers]{exam}
\usepackage{amsfonts,amssymb,amsmath,amsthm}
\usepackage[hidelinks]{hyperref}
\usepackage{pgf,tikz,pgfplots}
\pgfplotsset{compat=1.15}
\usepgfplotslibrary{fillbetween}
\usepackage{mathrsfs}
\usepackage{marginnote}
\usetikzlibrary{arrows}
\usetikzlibrary{calc}


\pagestyle{headandfoot}

\firstpageheader{Date: \underline{\hspace{2.5in}}}{}{Name: \underline{\hspace{2.5in}}}
%\firstpageheadrule

\runningheader{Higher Level Mathematics Analysis \& Approaches}{}{Page \thepage\ of \numpages}
\runningheadrule

\firstpagefooter{}{}{}
\runningfooter{}{}{}

\begin{document}

    \begin{center}

        \fbox{\fbox{\parbox{6in}{

        This markscheme is confidential and for the exclusive use of examiners in this examination session.
        It is the property of Upper Canada College and must not be reproduced or distributed to any other person without the authorization of Mr. Damion Walker.xw

        }}}

        \vspace{3cm}

        \includegraphics{logo.png}

    \end{center}

    \vspace{2cm}

    \begin{questions} %------------------------------------------
    
        \fullwidth{
        Answer all questions in the area provided. Full marks are not necessarily awarded for a correct answer with no working. Answers must be supported by working and/or explanations. Solutions found from a graphic display calculator should be supported by suitable working. For example, if graphs are used to find a solution, you should sketch these as part of your answer. Where an answer is incorrect, some marks may be given for a correct method, provided this is shown by written working. You are therefore advised to show all working.}

        \newpage %---------------------------------------------------

        \question \label{DisplayModeExample}

        [Maximum Mark: \numpoints]
        \newline
        \newline
        \textbf{In this question you will explore the convergence of infinite sequences and a mathematical test for determining if a specific sequence converges.}

        \begin{parts}
            \begin{EnvUplevel}
                Consider the following sequence:
                \[
                    \frac{1}{2}, \frac{1}{4}, \frac{1}{8}, \dots
                \]
            \end{EnvUplevel}
            
            %%%%%%%%%%%%%%%%%%%%%%%%%%%%%%%%%%%%%%%%%%%%%%%%%%%%%%%%%%%%%%%%%%%%%%%%%%%%%%%%%%%%%%%%%%%%%%

            \part[1] Find the common ratio between consecutive terms of the sequence.

            \begin{solutionordottedlines}[3em]
                The common ratio is $\frac{1}{2}$, or 0.5
                \marginnote{A1}
            \end{solutionordottedlines}
            \vspace{1em}

            %%%%%%%%%%%%%%%%%%%%%%%%%%%%%%%%%%%%%%%%%%%%%%%%%%%%%%%%%%%%%%%%%%%%%%%%%%%%%%%%%%%%%%%%%%%%%%

            \part[1] Hence, find the 4\textsuperscript{th} and 5\textsuperscript{th} terms of the sequence.

            \begin{solutionordottedlines}[3em]
                The fourth and fifth terms of the sequence are $\frac{1}{16}$ and $\frac{1}{32}$, respectively.
                \marginnote{A1}
            \end{solutionordottedlines} 
            \vspace{1em}

            %%%%%%%%%%%%%%%%%%%%%%%%%%%%%%%%%%%%%%%%%%%%%%%%%%%%%%%%%%%%%%%%%%%%%%%%%%%%%%%%%%%%%%%%%%%%%%

            \part[1] Hence, state whether the sequence converges (i.e. approaches a finite value) or diverges (i.e. tends to infinity).

            \begin{solutionordottedlines}[3em]
                The sequence converges.
                \marginnote{A1}
            \end{solutionordottedlines}
            \vspace{1em}

            %%%%%%%%%%%%%%%%%%%%%%%%%%%%%%%%%%%%%%%%%%%%%%%%%%%%%%%%%%%%%%%%%%%%%%%%%%%%%%%%%%%%%%%%%%%%%%

            \part[1] Hence, state a general formula for the $k$\textsuperscript{th} element of the sequence.

            \begin{solutionordottedlines}[3em]
                One possible formula is
                \[
                    T_k = \left(\frac{1}{2}\right)^{k}
                    \marginnote{A1}
                \]
            \end{solutionordottedlines}
            \vspace{1em}

            %%%%%%%%%%%%%%%%%%%%%%%%%%%%%%%%%%%%%%%%%%%%%%%%%%%%%%%%%%%%%%%%%%%%%%%%%%%%%%%%%%%%%%%%%%%%%%

            \part[1] Find a general formula for the following sequence:
            \[
                1, 2, 4, 8\dots
            \]

            \begin{solutionordottedlines}[3em]
                One possible formula is
                \[
                    T_k = 2^{k - 1}
                    \marginnote{A1}
                \]
            \end{solutionordottedlines}
            \vspace{1em}

            %%%%%%%%%%%%%%%%%%%%%%%%%%%%%%%%%%%%%%%%%%%%%%%%%%%%%%%%%%%%%%%%%%%%%%%%%%%%%%%%%%%%%%%%%%%%%%
            
            \newpage
            \part[1] Hence, state whether the sequence converges or diverges.

            \begin{solutionordottedlines}[3em]
                The sequence diverges.
                \marginnote{A1}
            \end{solutionordottedlines} 
            \vspace{1em}

            %%%%%%%%%%%%%%%%%%%%%%%%%%%%%%%%%%%%%%%%%%%%%%%%%%%%%%%%%%%%%%%%%%%%%%%%%%%%%%%%%%%%%%%%%%%%%%

            \part[1] Calculate the ratio between consecutive terms for terms 1 to 5 in the following sequence in decimal:
            \[
                1, 4, 9, 16\dots
            \]
            where $T_k = k^{2}$

            \begin{solutionordottedlines}[5em]
                The ratios are as follows:
                \[
                    4, 2.25, 1.777\dots, 1.5625
                    \marginnote{A1}
                \]
            \end{solutionordottedlines}
            \vspace{1em}

            %%%%%%%%%%%%%%%%%%%%%%%%%%%%%%%%%%%%%%%%%%%%%%%%%%%%%%%%%%%%%%%%%%%%%%%%%%%%%%%%%%%%%%%%%%%%%%

            \part[2] Show that the equation $y = \frac{(x + 1)^{2}}{x^{2}}$ has a horizontal asymptote at $y = 1$

            \begin{solutionordottedlines}[13em]
                \begin{align}
                    y &= \frac{(x + 1)^{2}}{x^{2}} \\
                    &= \frac{x^{2} + 2x + 1}{x^{2}} \\
                    &= 1 + \frac{2}{x} + \frac{1}{x^{2}}
                    \marginnote{M1}
                \end{align}
                Since $\frac{2}{x}$ and $\frac{1}{x^{2}}$ both have asymptotes at $y = 0$, then $1 + \frac{2}{x} + \frac{1}{x^{2}}$ has an asymptote at
                $y = 1$.
                \marginnote{R1}
            \end{solutionordottedlines}
            \vspace{1em}

            %%%%%%%%%%%%%%%%%%%%%%%%%%%%%%%%%%%%%%%%%%%%%%%%%%%%%%%%%%%%%%%%%%%%%%%%%%%%%%%%%%%%%%%%%%%%%%

            \part[1] Hence, state what the ratio between terms approaches in the sequence in \hyperref[part@1@7]{(g)}

            \begin{solutionordottedlines}[1.5cm]
                \marginnote{A1} The ratio between terms approaches 1.
            \end{solutionordottedlines}
            \vspace{1em}

            %%%%%%%%%%%%%%%%%%%%%%%%%%%%%%%%%%%%%%%%%%%%%%%%%%%%%%%%%%%%%%%%%%%%%%%%%%%%%%%%%%%%%%%%%%%%%%

            \begin{EnvUplevel}
                Consider the sequence defined by the following equation:
                \[
                    T_k = 2^{k} + \sin\left(\frac{\pi k}{2}\right)
                \]
            \end{EnvUplevel}

            \part[1] Question
            
        \end{parts}

    \end{questions}

\end{document}