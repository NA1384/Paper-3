\documentclass[11pt,addpoints,answers]{exam}
\usepackage{amsfonts,amssymb,amsmath, amsthm}
\usepackage{graphicx}
\usepackage{systeme}
\usepackage{pgf,tikz,pgfplots}
\pgfplotsset{compat=1.15}
\usepgfplotslibrary{fillbetween}
\usepackage{mathrsfs}
\usepackage{marginnote}
\usetikzlibrary{arrows}
\usetikzlibrary{calc}


\pagestyle{headandfoot}

\firstpageheader{Date: \underline{\hspace{2.5in}}}{}{Name: \underline{\hspace{2.5in}}}
%\firstpageheadrule

\runningheader{Higher Level Mathematics Analysis \& Approaches}{}{Page \thepage\ of \numpages}
\runningheadrule

\firstpagefooter{}{}{}
\runningfooter{}{}{}


\begin{document}

\begin{center}





\fbox{\fbox{\parbox{6in}{

This markscheme is confidential and for the exclusive use of examiners in this examination session.
It is the property of Upper Canada College and must not be reproduced or distributed to any other person without the authorization of Mr. Damion Walker.xw

}}}


\vspace{3cm}

\includegraphics{logo.png}

\end{center}

\vspace{2cm}

\begin{questions} %------------------------------------------
 
\fullwidth{
Answer all questions in the area provided. Full marks are not necessarily awarded for a correct answer with no working. Answers must be supported by working and/or explanations. Solutions found from a graphic display calculator should be supported by suitable working. For example, if graphs are used to find a solution, you should sketch these as part of your answer. Where an answer is incorrect, some marks may be given for a correct method, provided this is shown by written working. You are therefore advised to show all working.}

\newpage %---------------------------------------------------

\question \label{DisplayModeExample}

[Maximum Mark: \numpoints]
\newline
\newline
\textbf{In this question you will explore the convergence of infinite sequences and the mathematical test for determining if a specific sequence converges.}
\\
\begin{parts}

\part[1] Calculate the first 5 terms of the infinite sequence:

\[e^{-n}, {n \in \mathbb{N}} \]

\begin{solutionordottedlines}[2cm]
\marginnote{A1}
Using a GDC: 0.368, 0.135, 0.0498, 0.018, 0.00674.
\end{solutionordottedlines} 
\vspace{0.5cm}
\part[1] Hence, state what value the sequence approaches as \emph{n} approaches infinity:

\begin{solutionordottedlines}[1.5cm]
\marginnote{A1} The sequence approaches 0.
\end{solutionordottedlines} 
\vspace{0.5cm}
\part[2] By taking the limit of the sequence to infinity, support the conclusion deduced above and state whether the sequence converges or diverges.

\begin{solutionordottedlines}[7cm]
\[=\lim_{n\to\infty}{e^{-n}}\]
\marginnote{M1}
\[=\frac{\lim_{n\to\infty}{1}}{\lim_{n\to\infty}e^{n}}\]

Substituting \emph{n} for infinity provides:
\\
\[=\frac{1}{\infty}\]

\[=0\]

\marginnote{AG}
Therefore, the sequence converges at 0.
\end{solutionordottedlines}
\vspace{0.5cm}
\part[1] Calculate the first 5 terms of the infinite sequence:
\[e^{n}, {n \in \mathbb{N}} \]
\begin{solutionordottedlines}[2cm]
\marginnote{A1}
Using a GDC: 2.72, 7.39, 20.1, 54.6, 148.
\end{solutionordottedlines}
\vspace{0.5cm}
\part[1] Hence, state what value the sequence approaches as \emph{n} approaches infinity:

\begin{solutionordottedlines}[1.5cm]
\marginnote{A1} The sequence approaches infinity.
\end{solutionordottedlines} 
\vspace{0.5cm}
\part[1] By taking the limit of the sequence to infinity, support the conclusion deduced above and state whether the sequence converges or diverges.

\begin{solutionordottedlines}[7cm]
\[=\lim_{n\to\infty}{e^{n}}\]
\marginnote{M1}
Substituting \emph{x} for infinity provides:
\marginnote{M1}
\[=\infty\]
\marginnote{AG}
Therefore, the sequence diverges.
\end{solutionordottedlines}
\vspace{0.5cm}
\part[1] Calculate the first 5 terms of the infinite sequence:
\[\frac{n+1}{2n^2}, {n \in \mathbb{N}} \]
\begin{solutionordottedlines}[2cm]
\marginnote{A1}
Using a GDC: 1, 0.375, 0.222, 0.156, 0.12.
\end{solutionordottedlines}
\vspace{0.5cm}
\part[1] Hence, state what value the sequence approaches as \emph{n} approaches infinity:

\begin{solutionordottedlines}[1.5cm]
\marginnote{A1} The sequence approaches 0.
\end{solutionordottedlines} 
\vspace{0.5cm}
\part[2] By taking the limit of the sequence to infinity, support the conclusion deduced above and state whether the sequence converges or diverges.

\begin{solutionordottedlines}[7cm]
\[=\lim_{n\to\infty}\frac{n+1}{2n^2}\]
\marginnote{M1}
\[=\frac{1}{2}*(\lim_{n\to\infty}\frac{1}{n}+\lim_{n\to\infty}\frac{1}{n^{2}})\]
\marginnote{M1}
Substituting \emph{n} for infinity provides:
\\
\[=\frac{1}{2}*(0+0)\]

\[=0\]
\marginnote{AG}
Therefore, the sequence converges.
\end{solutionordottedlines}
\vspace{0.5cm}
\part[1] Calculate the first 5 terms of the infinite sequence:
\[\frac{2n}{\ln(n+1)}, {n \in \mathbb{N}} \]
\begin{solutionordottedlines}
\marginnote{A1}
Using a GDC: 2.89, 3.64, 4.33, 4.97, 5.58.
\end{solutionordottedlines}
\vspace{0.5cm}
\part[1] Hence, state what value the sequence approaches as \emph{n} approaches infinity:

\begin{solutionordottedlines}[1.5cm]
\marginnote{A1} The sequence approaches infinity.
\end{solutionordottedlines} 
\vspace{0.5cm}
\part[2] By taking the limit of the sequence to infinity, support the conclusion deduced above and state whether the sequence converges or diverges.

\begin{solutionordottedlines}[7cm]
\[=\lim_{n\to\infty}\frac{2n}{\ln(n+1)}\]
\marginnote{M1}
Using L'Hôpital's rule:
\[=2*(\lim_{n\to\infty}\frac{1}{\frac{1}{n+1}})\]
\marginnote{M1}
\[=2*\lim_{n\to\infty}(n+1)\]
\\
Substituting \emph{n} for infinity provides:
\\
\[=2*\infty\]
\[=\infty\]
\marginnote{AG}
Therefore, the sequence diverges.
\end{solutionordottedlines}
\vspace{0.5cm}
\part[8] For a positive infinite sequence $\displaystyle a_{n}$, the ratio test for convergence is defined by calculating $\displaystyle L = \lim_{n\to\infty}\frac{a_{n+1}}{a_{n}}$. Hence, calculate \emph{L} for the sequences previously investigated.

\begin{solutionordottedlines}[10cm]
\\
\[L_{1} = \lim_{n\to\infty}\frac{e^{-(n+1)}}{e^{-n}}\]
\marginnote{M1}
\[L_{1} = \frac{\lim_{n\to\infty}{e^{-(n+1)}}}{\lim_{n\to\infty}e^{-n}}\]
\\
\[L_{1} = \frac{\lim_{n\to\infty}{e^{-(n+1)}}}{\lim_{n\to\infty}e^{-n}}\]

\marginnote{A1}
\[L_{1}=\]
\marginnote{M1}

\marginnote{A1}
\[L_{2}=\]
\marginnote{M1}

\marginnote{A1}
\[L_{3}=\]
\marginnote{M1}

\marginnote{A1}
\[L_{4}=\]
\end{solutionordottedlines} 
\vspace{0.5cm}

\part[3] Hence, state the possible conclusions that can be deduced from the results of a ratio test and the possible values of \emph{L}.
\begin{solutionordottedlines}[4cm]
\vspace{0.25cm}
\newline
\marginnote{A1}
If $L < 1$, the sequence converges.
\newline
\marginnote{A1}
If $L > 1$, the sequence diverges.
\newline
\marginnote{A1}
If $L = 1$ or undefined, the ratio test is inconclusive.
\end{solutionordottedlines}


\end{parts}


\end{questions}

\end{document}